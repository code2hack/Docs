\documentclass{article}
\usepackage{amsmath}
\usepackage{pdfpages}
\usepackage{xeCJK}
\usepackage{hyperref}
\usepackage{listings}
\lstset{
 columns=fixed,       
 numbers=left,                                        % 在左侧显示行号
 numberstyle=\tiny\color{gray},                       % 设定行号格式
 frame=single,                                          % 不显示背景边框
 keywordstyle=\color[RGB]{40,40,255},                 % 设定关键字颜色
 numberstyle=\footnotesize\color{darkgray},           
 commentstyle=\it\color[RGB]{0,96,96},                % 设置代码注释的格式
 stringstyle=\rmfamily\slshape\color[RGB]{128,0,0},   % 设置字符串格式
 showstringspaces=false,                              % 不显示字符串中的空格
 language=c++,                                        % 设置语言
}
\newcommand{\page}[2]{\hyperlink{#1.pdf.#2}{link}}



\title{\vspace{-4cm}周工作汇报}
\author{陈辉}
\begin{document}
\maketitle
\section{本周工作}
\subsection*{文献阅读}
\begin{enumerate}
\item \textit{Near-Data Processing for Machine Learning}\\
NDP方法在机器学习中的应用探索,主要思想是将SGD算法部署在智能SSD上.
\item \textit{Evolution Strategies as a Scalable Alternative to Reinforcement Learning}\\
OpenAI小组将进化策略应用到增强学习中进行训练,主要特点
\begin{itemize}
\item 可并行性极佳
\item 训练效率高
\item 鲁棒性好
\\
(这篇文章难度较大,还未深入)

\end{itemize}

\end{enumerate}
\section{下步工作计划}
主要思路:\\
当前主流使用GPU对神经网络进行训练,对于深度神经网络、海量训练集的训练效率不理想,瓶颈可能在于(直观分析有待考证):1.受限于GPU显存限制,需要频繁的GPU-内存-硬盘的数据交换。2.简单地叠加GPU阵列并不能有效地提升训练效率.\\
尝试将OpenAI提出的进化策略(或者考虑遗传算法)部署在存储端进行。例如对CPU-FPGA(arm)-SSD这样的架构进行大规模扩展,如果最终性能可以接近于线性提升,我认为可以取代目前CPU-GPU的模式。

\subsection{机器学习方向}
\begin{enumerate}
\item 学习进化策略、遗传算法等,搜集相关资料研究将其应用于当前主流神经网络训练的可行性
\item 学习当前随机梯度下降算法与变种 
\end{enumerate}

\subsection{NDP方向}
\begin{enumerate}
\item 进一步阅读有关Near-Data Processing和In-Storage Compute的文章
\item 搜集资料以进一步分析基于NDP作海量数据训练的可行性
\item 寻找是否有较成熟的NDP平台架构
\item 复习体系结构相关知识
\end{enumerate}
%\includepdf[pages=-]{.pdf}
\end{document}
