              
\documentclass[xelatex]{beamer}
%
% Choose how your presentation looks.
%
% For more themes, color themes and font themes, see:
% http://deic.uab.es/~iblanes/beamer_gallery/index_by_theme.html
%
\mode<presentation>
{
  \usetheme{default}      % or try Darmstadt, Madrid, Warsaw, ...
  \usecolortheme{default} % or try albatross, beaver, crane, ...
  \usefonttheme{default}  % or try serif, structurebold, ...
  \setbeamertemplate{navigation symbols}{}
  \setbeamertemplate{caption}[numbered]
} 

\usepackage[english]{babel}
%\usepackage[utf8x]{inputenc}
\usepackage{xeCJK}
\usepackage{verbatim}

\title[Your Short Title]{基于GlusterFS的数据分布策略研究}
\author{陈辉}
%\institute{Where You're From}
\date{}

\begin{document}

\begin{frame}
  \titlepage
\end{frame}

% Uncomment these lines for an automatically generated outline.
\begin{frame}{Outline}
  \tableofcontents
\end{frame}
\section{Introduction}
\begin{frame}{概述}
本研究致力于在分布式(或高性能计算)环境下,针对不同工作负载进行冷热数据动态负载均衡。主要思想是:
\begin{itemize}
\item 对用户程序实时监控,抽取与IO操作相关性较大的行为特征;
\item 根据用户IO读写历史对文件系统中的数据进行聚类;
\item 将用户、用户程序特征与聚类后的数据进行关联;并根据访问时序对聚类后的数据进行粗略排序。
\item 针对正在运行或即将运行的用户程序特征,将关联的数据(主要针对小文件)按时序加载至SSD.
\end{itemize}
\end{frame}



\section{GlusterFS简介}

\begin{frame}{GlusterFS简介}

%\vskip 1cm
\begin{block}{GlusterFS架构简述}
Some examples of commonly used commands and features are included, to help you get started.
\end{block}

\begin{block}{GlusterFS tiering模块}
\end{block}
\end{frame}

\section{数据采集(监控)}
\subsection{日志分析}
\begin{frame}{日志分析}
日志分析能够比较直接地比对分析用户程序运行与系统层IO活动的逻辑关联性.
%此处应有图

\end{frame}





\begin{comment}
% Commands to include a figure:
%\begin{figure}
%\includegraphics[width=\textwidth]{your-figure's-file-name}
%\caption{\label{fig:your-figure}Caption goes here.}
%\end{figure}

\begin{table}
\centering
\begin{tabular}{l|r}
Item & Quantity \\\hline
Widgets & 42 \\
Gadgets & 13
\end{tabular}
\caption{An example table.}
\end{table}
\end{frame}
\end{comment}

\end{document}
