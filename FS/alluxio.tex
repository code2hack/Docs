\documentclass{article}
\usepackage{amsmath}
\usepackage{pdfpages}
\usepackage{verbatim}
\usepackage{xeCJK}
\usepackage{hyperref}
\usepackage{listings}
\lstset{
 columns=fixed,       
 numbers=left,                                        % 在左侧显示行号
 numberstyle=\tiny\color{gray},                       % 设定行号格式
 frame=single,                                          % 不显示背景边框
 keywordstyle=\color[RGB]{40,40,255},                 % 设定关键字颜色
 numberstyle=\footnotesize\color{darkgray},           
 commentstyle=\it\color[RGB]{0,96,96},                % 设置代码注释的格式
 stringstyle=\rmfamily\slshape\color[RGB]{128,0,0},   % 设置字符串格式
 showstringspaces=false,                              % 不显示字符串中的空格
 language=c++,                                        % 设置语言
}
\hypersetup{
    colorlinks=true,
    linkcolor=blue,
    filecolor=magenta,
    urlcolor=cyan,
}
\newcommand{\paper}[2]{\hyperlink{./papers/#1.pdf.#2}{(P#2)}}
\newcommand{\book}[2]{\href[page=#2]{./books/#1.pdf}{(P#2)}}


\title{Alluxio}
\author{陈辉}
\date{}
\begin{document}
\maketitle
\tableofcontents
\newpage
\section{Introduction}

\subsection{Challenges with the Existing Ecosystem}
\begin{itemize}
    \item Work Duplication;
    \item Data Duplication;
    \item Performance;
    \item Management and Administration.
\end{itemize}

%\subsection{Virtual Distributed File System}
%VDFS mainly consists of \textbf{NAPIs} and \textbf{SAPIs}
%\subsection{Results and Impacts Highlights}
\setcounter{subsection}{3}
\subsection{Alluxio Ecosystem}
\subsection{Contributions}

\section{Lineage in VDFS}
Lineage circumvents the limitations of replication by \textbf{re-excuting the operations that created the output.}
\par
Challenges:
\begin{itemize}
        \item \textit{bounding the recomputation cost for a long-running storage system}.
        \item \textit{how to allocate resources for recomputation}
\end{itemize}


\end{document}

\includepdf[link=true,pages=-,fitpaper]{./papers/<++>.pdf}
%\includepdf[link=true,pages=-,fitpaper]{./papers/<++>.pdf}
%\includepdf[link=true,pages=-,fitpaper,angle=-90]{./papers/<++>.pdf}
