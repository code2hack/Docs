\documentclass{article}
\usepackage{amsmath}
\usepackage{amssymb}
\usepackage{dsfont}
\usepackage{unicode-math}
\numberwithin{equation}{section}
\numberwithin{figure}{section}
\usepackage{graphicx}
\graphicspath{{./images/}}

\usepackage{pdfpages}
\usepackage{verbatim}
\usepackage{xeCJK}
\usepackage{hyperref}
%\usepackage{apacite}
\usepackage{natbib}
\usepackage{grffile}
\usepackage{listings}
\lstset{
 columns=fixed,       
 numbers=left,                                        % 在左侧显示行号
 numberstyle=\tiny\color{gray},                       % 设定行号格式
 frame=single,                                          % 不显示背景边框
 keywordstyle=\color[RGB]{40,40,255},                 % 设定关键字颜色
 numberstyle=\footnotesize\color{darkgray},           
 commentstyle=\it\color[RGB]{0,96,96},                % 设置代码注释的格式
 stringstyle=\rmfamily\slshape\color[RGB]{128,0,0},   % 设置字符串格式
 showstringspaces=true,                              % 显示字符串中的空格
 language=c++,                                        % 设置语言
}
\hypersetup{
    colorlinks=true,
    linkcolor=blue,
    filecolor=magenta,
    urlcolor=cyan,
}
%--------Commands for local pdf reference------%
\newcommand{\citeint}[2]{\cite{#1}\hyperlink{./references/#1.pdf.#2}{(P#2)}}
\newcommand{\citeext}[2]{\href[page=#2]{../references/#1.pdf}{(P#2)}}
\newcommand{\citeinclude}[1]{\includepdf[link=true,pages=-,fitpaper=true]{./references/#1.pdf}}
\newcommand{\book}[1]{\citeext{TLPI}{#1}}
%--------Commands/Operators for math------%
\DeclareMathOperator*{\argmax}{argmax}
\DeclareMathOperator*{\argmin}{argmin}
%--------Environments for math------------%
\newcounter{topic}[subsection]
\newenvironment{defi}{\refstepcounter{topic}\label{\thesubsection.\thetopic}\par\medskip
   \noindent \textbf{Definition~\thesubsection.\thetopic~} \rmfamily }{ \medskip}
\newenvironment{theo}{\refstepcounter{topic}\label{\thesubsection.\thetopic}\par\medskip
   \noindent \textbf{Theorem~\thesubsection.\thetopic~} \rmfamily }{\medskip}
\newenvironment{coro}{\refstepcounter{topic}\label{\thesubsection.\thetopic}\par\medskip
   \noindent \textbf{Corollary~\thesubsection.\thetopic~} \rmfamily }{ \medskip}
\newenvironment{lemm}{\refstepcounter{topic}\label{\thesubsection.\thetopic}\par\medskip
   \noindent \textbf{Lemma~\thesubsection.\thetopic~} \rmfamily }{ \medskip}
\newenvironment{exam}[1]{\refstepcounter{topic}\label{\thesubsection.\thetopic}\par\medskip
   \noindent \textbf{Example~\thesubsection.\thetopic~} \rmfamily }{ \medskip}

\newenvironment{rema}[1]{\refstepcounter{topic}\label{\thesubsection.\thetopic}\par\medskip
   \noindent \textbf{Remark~\thesubsection.\thetopic~} \rmfamily }{ \medskip}
\newenvironment{prof}{\par\medskip \noindent \textit{Proof.}~\rmfamily}{\medskip}

\title{The Linux Programming Interfaces}
\author{陈辉}
\date{}
\begin{document}
\maketitle
\tableofcontents
\newpage
\section{History and Standards}
\section{Fundamental Concepts}
\subsection{The Kernel}
\subsection{The Shell}
\subsection{Users and Groups}
\subsubsection*{Users}
\begin{itemize}
\item \textit{login name}
\item \textit{user ID(UID)}
\item \textit{Group ID}
\item \textit{Home directory}
\item \textit{Login shell}:the name of the program to be executed to interpret user commands.
\end{itemize}
These information of each user resides in \textit{password files}
\subsubsection*{Groups}
\subsubsection*{Superuser}
userID = 0.
\subsection{Single Directory Hierarchy, Directories, Links, and Files}
Figure of single directory hierarchy.\book{71}
\subsubsection*{FIle Types}
\textit{The other file types}:\\
devices, pipes, sockets, directories, and symbolic links.
\subsubsection*{Directories and links}
The links between directories establish the directory hierarchy.
\subsubsection*{Symbolic links}
\subsubsection*{Pathnames}
\begin{itemize}
\item \textit{absolute pathname}
\item \textit{relative pathname}
\end{itemize}

\subsection{File I/O Model}
\par\textit{universality of I/O}:The same system calls(\textit{open,read,write,close}) are used to perform I/O on \textbf{all types of files}.
\subsubsection*{File descriptors}
A nonnegative integer obtained by a call to \textit{open()}.\\
Often 0 for input, 1 for output and 2 for errors or other abnormal messages.
\subsection{Programs}
\subsubsection*{Filters}
\subsubsection*{Command-line arguments}

\subsection{Processes}
\subsubsection*{Process memory layout}
\textit{segments}:
\begin{itemize}
\item \textit{Text}
\item \textit{Data}
\item \textit{Heap}
\item \textit{Stack}
\end{itemize}

\subsubsection*{Process creation and execution}
\textit{fork()}:\\
The kernel creates the child process by making a duplicate of the parent process which inherits copies of parent's \textbf{data, stack, heap}. The text is placed in memory marked read-only shared by them.\\
\textit{execve()}:\\
child call execve() system calls to replace the origin segments with new target program.

\subsubsection*{Process ID and parent process ID}
\subsubsection*{Process termination and termination status}
child call \textit{\_exit()} or be killed by a signal.\\
parent \textit{wait()} for child's \textit{termination status}.
\subsubsection*{Process user and group identifiers}
\begin{itemize}
\item \textit{Real user ID and real group ID}
\item \textit{Effective user ID and effective group ID}
\item \textit{Supplementary group IDs}
\end{itemize}

\subsubsection*{The \textit{init} process}
The \textit{init} is the parent of all processes with a constant PID = 1.\\

\subsubsection*{Daemon processes}
background
\subsubsection*{Environment list}

\subsection{Memory Mappings}
\textit{mmap()}:




\subsection{Static and Shared Libraries}
\subsubsection*{Static libraries}
A static library is essentially a structured bundle of compiled object modules. To use functions from it we specify that library in the \textbf{link} command used to \textbf{build} a program.
\subsubsection*{Shared libraries}
Shared libraries were designed to address the wasting problems with static libraries.
\begin{itemize}
\item While building: the linker writes a record into the executable.
\item While runtime: \textit{dynamic linker} ensures the required shared libraries are found and loaded to the memory.
\end{itemize}

\subsection{Interprocess Communication(IPC) and Synchronization}
The set of mechanisms for interprocess communication(IPC):
\begin{itemize}
\item \textit{signals}
\item \textit{pipes}
\item \textit{sockets}
\item \textit{file locking}
\item \textit{message queues}
\item \textit{semaphores}
\item \textit{shared memory}
\end{itemize}

\subsection{Signals}
Signals are often described as ``software interrupts''
\subsection{Threads}
Threads \textbf{share}
\begin{itemize}
\item Text(for program codes).
\item Data.
\item Heap.
\item virtual memory.
\item global variables.(for communication)
\end{itemize}
Threads \textbf{differ in}
\begin{itemize}
\item Stack.
\item local variables.
\item function call linkage information(why?).
\end{itemize}
Threads' \textbf{advantages}
\begin{itemize}
\item easy to share data rather than multiprocesses.
\item good for parallel processing
\end{itemize}

\subsection{Process Groups and Shell Job Control}
\subsection{Sessions, Controlling Terminals, and Controlling Processes}
\subsection{Pseudoterminals}
A pseudoterminal is a \textbf{pair of connected virtual devices} known as \textbf{master and slave}.\\
Master drives the user program and slave drives terminal-oriented program. This connection is like a bridge. e.g. \textit{telnet and ssh}.
\subsection{Date and Time}
\begin{itemize}
\item \textit{Realtime}.
\item \textit{Process time}.
\end{itemize}

\subsection{Client-Server Architecture}
\subsection{Realtime}
\href{https://www.cl.cam.ac.uk/~mgk25/linux-posix.1b.txt}{POSIX.1b}:
\subsection{The /proc File System}
A virtual file system that provide an \textbf{interface to kernel data structures}.

\section{System Programming Concepts}

\subsection{System Calls}
\book{87}
\subsection{Library Functions}
\subsection{The standard C Library; The GNU C Library(glibc)}
\subsection{Handling Errors from System Calls and Library Functions}

\subsection{Notes on the Example Programs in this book}

\subsubsection{Command-Line Options and Arguments}
\subsubsection{Common Functions and Header Files}
\book{95}

\subsection{Portability Issues}

\section{File I/O}
Keypoints:
\begin{itemize}
\item System call APIs to perform file I/O.
\item File descriptor.
\end{itemize}

\subsection{Overview}
Key word:\textit{file descriptor}


\setcounter{section}{40}
\section{Fundamentals of Shared Libraries}

\subsection{Object Libraries}
\book{833}
\textbf{Object Library}: A set of object files.\\
\subsection{Static Libraries(archives)}
\subsubsection*{Creating and maintaining a static library}
Syntax:
\begin{lstlisting}
\$ ar \textit{options archive object-files}
\end{lstlisting}
\textbf{Conventional form of static libraries}: lib\textit{name}.a.
\begin{lstlisting}
	\$ gcc -g -c mod1.c mod2.c
	\$ ar r libdemo.a mod1.o mod2.o
	\$ rm mod1.o mod2.o
\end{lstlisting}
Delete modules from the archive:
\begin{lstlisting}
	\$ ar d libdemo.a mod2.o
\end{lstlisting}

\subsubsection*{Using a static library}
Basic way:
\begin{lstlisting}
\$ gcc -g -c prog.c
\$ gcc -g -o prog prog.o libdemo.a
\end{lstlisting}
Alternative way:
\begin{lstlisting}
\$ gcc -g -o prog prog.o -ldemo
\end{lstlisting}
'-ldemo' means '-l' and archive without lib prefix and .a suffix which resides in one of the standard directories(e.g. /usr/lib)

\subsection{Overview of Shared Libraries}
\subsection{Creating and Using Shared Libraries}
\textbf{ELF(Executable and Linking Format shared libraries)}:ELF is the format employed for executables and shared libraries on modern linux.
\subsubsection{Creating a Shared Library}
\begin{lstlisting}
\$ gcc -g -c -fPIC -Wall mod1.c mod2.c mod3.c
\$ gcc -g -o -shared -o libfoo.so mod1.o mod2.o mod3.o
\end{lstlisting}
\textbf{Remark.} Unlike static libraries, object modules cannot be add or deleted from previously built shared library(why?)\\
\subsubsection{Position-Independent Code}
\textit{-fPIC} specifies that the compiler generate position-independent code.
How to check whether an object file has been compiled with \textit{-fPIC}\book{882}

\subsubsection{Using a Shared Library}
Two main steps to utilize shared library:
\begin{itemize}
\item Embedding the name of the shared library inside the executable during linking.
\item Resolving the embedding library name: performed by \textit{dynamic linker}, named \textbf{/lib/ld-linux.so.2}.
\end{itemize}
\textbf{The LD\_LIBRARY\_PATH} enveironment variable. \book{840}

\subsubsection{The Shared Library Soname}
\book{884}

\setcounter{section}{5}
\section{Processes}
\subsection{Processes and Programs}
\href{https://en.wikipedia.org/wiki/Executable\_and\_Linkable\_Format}{Executable and Linking Format}
\subsection{Process ID and Parent Process ID}
\subsection{Memory Layout of a Process}
\textit{segments}:\\
The memory allocated to each process. Including:
\begin{itemize}
        \item \textit{text segment}: Machine-language instructions. Read-only, shared. \\
        \item \textit{Initialized data segment}: global and static variables explicitly initialized. \\
        \item \textit{Uninitialized data segment}. \\
        \item \textit{Stack}. One stack frame for each current called function. \\
        \item \textit{Heap}: dynamically allocated data.
\end{itemize}
\subsection{Virtual Memory Management}
Figure: Typical memory layout of a process.\book{163}\\
\textcolor{green}{\textit{pages}}:\\
A virtual memory scheme splits the memory used by each program into small fixed-size units called pages. \\
\textit{page table}: \book{164} \\
Processes can use \textit{shmget()} and \textit{mmap()} to explicitly request sharing of memory regions with other processes, for the purpose of \textcolor{red}{interprocess communication}. \\

\subsection{The Stack and Stack Frames}
\href{http://www.cs.uwm.edu/classes/cs315/Bacon/Lecture/HTML/ch10s07.html}{See this lecture} \\


\end{document}

%\includepdf[link=true,pages=-,fitpaper]{./papers/<++>.pdf}
%\includepdf[link=true,pages=-,fitpaper,angle=-90]{./papers/<++>.pdf}
