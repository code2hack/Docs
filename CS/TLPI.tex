\documentclass{article}
\usepackage{amsmath}
\usepackage{pdfpages}
\usepackage{verbatim}
\usepackage{xeCJK}
\usepackage{hyperref}
\usepackage{listings}
\lstset{
 columns=fixed,       
 numbers=left,                                        % 在左侧显示行号
 numberstyle=\tiny\color{gray},                       % 设定行号格式
 frame=single,                                          % 不显示背景边框
 keywordstyle=\color[RGB]{40,40,255},                 % 设定关键字颜色
 numberstyle=\footnotesize\color{darkgray},           
 commentstyle=\it\color[RGB]{0,96,96},                % 设置代码注释的格式
 stringstyle=\rmfamily\slshape\color[RGB]{128,0,0},   % 设置字符串格式
 showstringspaces=false,                              % 不显示字符串中的空格
 language=c++,                                        % 设置语言
}
\newcommand{\paper}[2]{\hyperlink{./papers/#1.pdf.#2}{(P#2)}}
\newcommand{\book}[1]{\href[page=#1]{./books/TLPI.pdf}{(P#1)}}


\title{The Linux Programming Interfaces}
\author{陈辉}
\date{}
\begin{document}
\maketitle
\tableofcontents
\newpage
\section{History and Standards}
\section{Fundamental Concepts}
\subsection{The Kernel}
\subsection{The Shell}
\subsection{Users and Groups}
\subsubsection*{Users}
\begin{itemize}
\item \textit{login name}
\item \textit{user ID(UID)}
\item \textit{Group ID}
\item \textit{Home directory}
\item \textit{Login shell}:the name of the program to be executed to interpret user commands.
\end{itemize}
These information of each user resides in \textit{password files}
\subsubsection*{Groups}
\subsubsection*{Superuser}
userID = 0.
\subsection{Single Directory Hierarchy, Directories, Links, and Files}
Figure of single directory hierarchy.\book{71}
\subsubsection*{FIle Types}
\textit{The other file types}:\\
devices, pipes, sockets, directories, and symbolic links.
\subsubsection*{Directories and links}
The links between directories establish the directory hierarchy.
\subsubsection*{Symbolic links}
\subsubsection*{Pathnames}
\begin{itemize}
\item \textit{absolute pathname}
\item \textit{relative pathname}
\end{itemize}

\subsection{File I/O Model}
\par\textit{universality of I/O}:The same system calls(\textit{open,read,write,close}) are used to perform I/O on \textbf{all types of files}.
\subsubsection*{File descriptors}
A nonnegative integer obtained by a call to \textit{open()}.\\
Often 0 for input, 1 for output and 2 for errors or other abnormal messages.
\subsection{Programs}
\subsubsection*{Filters}
\subsubsection*{Command-line arguments}

\subsection{Processes}
\subsubsection*{Process memory layout}
\textit{segments}:
\begin{itemize}
\item \textit{Text}
\item \textit{Data}
\item \textit{Heap}
\item \textit{Stack}
\end{itemize}

\subsubsection*{Process creation and execution}
\textit{fork()}:\\
The kernel creates the child process by making a duplicate of the parent process which inherits copies of parent's \textbf{data, stack, heap}. The text is placed in memory marked read-only shared by them.\\
\textit{execve()}:\\
child call execve() system calls to replace the origin segments with new target program.

\subsubsection*{Process ID and parent process ID}
\subsubsection*{Process termination and termination status}
child call \textit{\_exit()} or be killed by a signal.\\
parent \textit{wait()} for child's \textit{termination status}.
\subsubsection*{Process user and group identifiers}
\begin{itemize}
\item \textit{Real user ID and real group ID}
\item \textit{Effective user ID and effective group ID}
\item \textit{Supplementary group IDs}
\end{itemize}

\subsubsection*{The \textit{init} process}
The \textit{init} is the parent of all processes with a constant PID = 1.\\

\subsubsection*{Daemon processes}
background
\subsubsection*{Environment list}

\subsection{Memory Mappings}
\textit{mmap()}:




\subsection{Static and Shared Libraries}
\subsubsection*{Static libraries}
A static library is essentially a structured bundle of compiled object modules. To use functions from it we specify that library in the \textbf{link} command used to \textbf{build} a program.
\subsubsection*{Shared libraries}
Shared libraries were designed to address the wasting problems with static libraries.
\begin{itemize}
\item While building: the linker writes a record into the executable.
\item While runtime: \textit{dynamic linker} ensures the required shared libraries are found and loaded to the memory.
\end{itemize}

\subsection{Interprocess Communication(IPC) and Synchronization}
The set of mechanisms for interprocess communication(IPC):
\begin{itemize}
\item \textit{signals}
\item \textit{pipes}
\item \textit{sockets}
\item \textit{file locking}
\item \textit{message queues}
\item \textit{semaphores}
\item \textit{shared memory}
\end{itemize}

\subsection{Signals}
Signals are often described as ``software interrupts''
\subsection{Threads}
Threads \textbf{share}
\begin{itemize}
\item Text(for program codes).
\item Data.
\item Heap.
\item virtual memory.
\item global variables.(for communication)
\end{itemize}
Threads \textbf{differ in}
\begin{itemize}
\item Stack.
\item local variables.
\item function call linkage information(why?).
\end{itemize}
Threads' \textbf{advantages}
\begin{itemize}
\item easy to share data rather than multiprocesses.
\item good for parallel processing
\end{itemize}

\subsection{Process Groups and Shell Job Control}
\subsection{Sessions, Controlling Terminals, and Controlling Processes}
\subsection{Pseudoterminals}
A pseudoterminal is a \textbf{pair of connected virtual devices} known as \textbf{master and slave}.\\
Master drives the user program and slave drives terminal-oriented program. This connection is like a bridge. e.g. \textit{telnet and ssh}.
\subsection{Date and Time}
\begin{itemize}
\item \textit{Realtime}.
\item \textit{Process time}.
\end{itemize}

\subsection{Client-Server Architecture}
\subsection{Realtime}
\href{https://www.cl.cam.ac.uk/~mgk25/linux-posix.1b.txt}{POSIX.1b}:
\subsection{The /proc File System}
A virtual file system that provide an \textbf{interface to kernel data structures}.

\section{System Programming Concepts}

\subsection{System Calls}
\book{87}
\subsection{Library Functions}
\subsection{The standard C Library; The GNU C Library(glibc)}
\subsection{Handling Errors from System Calls and Library Functions}

\subsection{Notes on the Example Programs in this book}

\subsubsection{Command-Line Options and Arguments}
\subsubsection{Common Functions and Header Files}
\book{95}

\subsection{Portability Issues}

\section{File I/O}
Keypoints:
\begin{itemize}
\item System call APIs to perform file I/O.
\item File descriptor.
\end{itemize}

\subsection{Overview}


\end{document}

%\includepdf[link=true,pages=-,fitpaper]{./papers/<++>.pdf}
%\includepdf[link=true,pages=-,fitpaper,angle=-90]{./papers/<++>.pdf}
