\documentclass{article}
\usepackage{amsmath}
\usepackage{amssymb}
\usepackage{dsfont}
\usepackage{unicode-math}
\numberwithin{equation}{section}
\numberwithin{figure}{section}
\usepackage{graphicx}
\graphicspath{{./images/}}

\usepackage{pdfpages}
\usepackage{verbatim}
\usepackage{xeCJK}
\usepackage{hyperref}
%\usepackage{apacite}
\usepackage{natbib}
\usepackage{grffile}
\usepackage{listings}
\lstset{
 columns=fixed,       
 numbers=left,                                        % 在左侧显示行号
 numberstyle=\tiny\color{gray},                       % 设定行号格式
 frame=single,                                          % 不显示背景边框
 keywordstyle=\color[RGB]{40,40,255},                 % 设定关键字颜色
 numberstyle=\footnotesize\color{darkgray},           
 commentstyle=\it\color[RGB]{0,96,96},                % 设置代码注释的格式
 stringstyle=\rmfamily\slshape\color[RGB]{128,0,0},   % 设置字符串格式
 showstringspaces=true,                              % 显示字符串中的空格
 language=c++,                                        % 设置语言
}
\hypersetup{
    colorlinks=true,
    linkcolor=blue,
    filecolor=magenta,
    urlcolor=cyan,
}
%--------Commands for local pdf reference------%
\newcommand{\citeint}[2]{\cite{#1}\hyperlink{./references/#1.pdf.#2}{(P#2)}}
\newcommand{\citeext}[2]{\href[page=#2]{./references/#1.pdf}{(P#2)}}
\newcommand{\citeinclude}[1]{\includepdf[link=true,pages=-,fitpaper=true]{./references/#1.pdf}}
\newcommand{\book}[1]{\citeext{Speech_and_Language_Processing}{#1}}
%--------Commands/Operators for math------%
\DeclareMathOperator*{\argmax}{argmax}
\DeclareMathOperator*{\argmin}{argmin}
%--------Environments for math------------%
\newcounter{topic}[subsection]
\newenvironment{defi}{\refstepcounter{topic}\label{\thesubsection.\thetopic}\par\medskip
   \noindent \textbf{Definition~\thesubsection.\thetopic~} \rmfamily }{ \medskip}
\newenvironment{theo}{\refstepcounter{topic}\label{\thesubsection.\thetopic}\par\medskip
   \noindent \textbf{Theorem~\thesubsection.\thetopic~} \rmfamily }{\medskip}
\newenvironment{coro}{\refstepcounter{topic}\label{\thesubsection.\thetopic}\par\medskip
   \noindent \textbf{Corollary~\thesubsection.\thetopic~} \rmfamily }{ \medskip}
\newenvironment{lemm}{\refstepcounter{topic}\label{\thesubsection.\thetopic}\par\medskip
   \noindent \textbf{Lemma~\thesubsection.\thetopic~} \rmfamily }{ \medskip}
\newenvironment{exam}[1]{\refstepcounter{topic}\label{\thesubsection.\thetopic}\par\medskip
   \noindent \textbf{Example~\thesubsection.\thetopic~} \rmfamily }{ \medskip}

\newenvironment{rema}[1]{\refstepcounter{topic}\label{\thesubsection.\thetopic}\par\medskip
   \noindent \textbf{Remark~\thesubsection.\thetopic~} \rmfamily }{ \medskip}
\newenvironment{prof}{\par\medskip \noindent \textit{Proof.}~\rmfamily}{\medskip}
\title{Notes for \textit{Speech and Language Processing}}
\author{Code2Hack}
\date{}
\begin{document}
\maketitle
\tableofcontents
\newpage


\section{Introduction}
\section{Regular Expressions etc.}
\book{10}
\subsection{Regular Expressions}
\subsubsection{Basic Regular Expression Patterns}

\begin{itemize}
        \item brackets: e.g [wW] means w or W. [a-z] means a to z. [\^{}a-z] not a upper case letter. \textbf{Notice,} only when \^{} is the first character in brackets it negates the pattern.\\
        \item ?: /colou?r/ = color or colour. The preceding character or nothing. \\

        \item Kleene star: /[ab]*/= 0 or more a's or b's. \\
        \item Kleene +: /[0-9]+/= a sequence of digits. \\
        \item .: any single character. /.*/= any string\\
        \item Anchors:\^{} start of a line. \$ = end. \textbackslash b = boundary.
\end{itemize}
\subsubsection{Disjunction, Grouping, Precedence}

\section{N-Gram Language Models}
\subsection{N-Grams}
Here we denote $w_1^{n-1}$ as the sequence $w_1, w_2, \dots, w_{n-1}$
\par
\begin{equation*}
        P(w_1^n) = \prod_{k=1}^n P(w_k | w_1^{k-1})
\end{equation*}
For N-gram model which fufills N-1th-order Markov Property, 
\begin{equation*}
        P(w_1^n) = \prod_{k=1}^n P(w_k | w_{k-N+1}^{k-1})
\end{equation*}

%\bibliographystyle{plain}
%\bibliography{overview_RL}

\end{document}


%\includepdf[link=true,pages=-,fitpaper,angle=-90]{./papers/<++>.pdf}
