\documentclass{article}
\usepackage{amsmath}
\usepackage{amssymb}
\usepackage{dsfont}
\usepackage{unicode-math}
\numberwithin{equation}{section}
\numberwithin{figure}{section}
\usepackage{graphicx}
\graphicspath{{./images/}}

\usepackage{pdfpages}
\usepackage{verbatim}
\usepackage{xeCJK}
\usepackage{hyperref}
%\usepackage{apacite}
\usepackage{natbib}
\usepackage{grffile}
\usepackage{listings}
\lstset{
 columns=fixed,       
 numbers=left,                                        % 在左侧显示行号
 numberstyle=\tiny\color{gray},                       % 设定行号格式
 frame=single,                                          % 不显示背景边框
 keywordstyle=\color[RGB]{40,40,255},                 % 设定关键字颜色
 numberstyle=\footnotesize\color{darkgray},           
 commentstyle=\it\color[RGB]{0,96,96},                % 设置代码注释的格式
 stringstyle=\rmfamily\slshape\color[RGB]{128,0,0},   % 设置字符串格式
 showstringspaces=true,                              % 显示字符串中的空格
 language=c++,                                        % 设置语言
}
\hypersetup{
    colorlinks=true,
    linkcolor=blue,
    filecolor=magenta,
    urlcolor=cyan,
}
%--------Commands for local pdf reference------%
\newcommand{\citeint}[2]{\cite{#1}\hyperlink{./references/#1.pdf.#2}{(P#2)}}
\newcommand{\citeext}[2]{\href[page=#2]{../references/#1.pdf}{(P#2)}}
\newcommand{\citeinclude}[1]{\includepdf[link=true,pages=-,fitpaper=true]{./references/#1.pdf}}
\newcommand{\book}[1]{\citeext{Introductory_Functional_Analysis_with_Applications}{#1}}
%--------Commands/Operators for math------%
\DeclareMathOperator*{\argmax}{argmax}
\DeclareMathOperator*{\argmin}{argmin}
%--------Environments for math------------%
\newcounter{topic}[subsection]
\newenvironment{defi}[1]{\refstepcounter{topic}\label{\thesubsection.\thetopic}\medskip
\noindent \textbf{Definition~\thesubsection.\thetopic~(#1)}\par \rmfamily }{ \medskip}

\newenvironment{theo}[1]{\refstepcounter{topic}\label{\thesubsection.\thetopic}\par\medskip
\noindent \textbf{Theorem~\thesubsection.\thetopic~(#1)} \rmfamily }{\medskip}

\newenvironment{coro}[1]{\refstepcounter{topic}\label{\thesubsection.\thetopic}\par\medskip
\noindent \textbf{Corollary~\thesubsection.\thetopic~(#1)} \rmfamily }{ \medskip}

\newenvironment{lemm}[1]{\refstepcounter{topic}\label{\thesubsection.\thetopic}\par\medskip
\noindent \textbf{Lemma~\thesubsection.\thetopic~(#1)} \rmfamily }{ \medskip}

\newenvironment{exam}[1]{\refstepcounter{topic}\label{\thesubsection.\thetopic}\par\medskip
   \noindent \textbf{Example~\thesubsection.\thetopic~} \rmfamily }{ \medskip}

\newenvironment{rema}[1]{
\noindent \textbf{Remark~(#1)} \rmfamily }{ \medskip}

\newenvironment{proo}{\par\medskip \noindent \textit{Proof.}~\rmfamily}{\medskip}



\title{Functional Analysis}
\author{Code2Hack}
\date{}
\begin{document}
\maketitle
\tableofcontents
\newpage
\section{Metric Spaces}
\subsection{Metric Space}

\book{18}
\begin{defi}{Metric space,metric}
A \textit{metric space} is a pair $(X,\textit{d})$ where X is a set and $d:X \times X \rightarrow R $ 
\begin{enumerate}
\item d is real-valued, finite and nonnegative.
\item $d(x,y)=0$ iff x=y.
\item $d(x,y)=d(y,x)$
\item $d(x,y) \leq d(x,z)+d(z,y)$ (\textbf{Triangle inequality}).
\end{enumerate}
\end{defi}
\textbf{Examples}\\
\setcounter{topic}{5}
\begin{exam}{Sequence space $l^{\infty}$}
\begin{itemize}
\item $x=(\xi_j)$
\item $|\xi_j| \leq c_x$, where $c_x$ is a real number may depend on x, but not on j.
\item $d(x,y)=\sup_{j \in N} |\xi_j - \eta_j|$
\end{itemize}
\end{exam}
\begin{exam}{Function space C[a,b]}
X is the set of all real-valued functions defined on closed interval $J=[a,b]$ and\\
\begin{equation*}
d(x,y)=\max_{t \in J} |x(t) - y(t)|,
\end{equation*}
\end{exam}

\subsection{Further Examples of Metric Spaces}
\book{24}
\begin{exam}{Sequence space s}
In contrast with \ref{1.1.6},
\end{exam}

\subsection{Open Set, Closed Set, Neighborhood}
\begin{defi}{Ball and sphere}
\end{defi}

\begin{defi}{Open set, closed set}
A subset M of a metric space is \textit{open} if it contains a ball about each of its points. A subset K is closed if $K^c = X - K$ is open.
\end{defi}

\begin{rema}
{Topological Space}
For the collection of all the open subsets of X called $\mathcal{T}$ :
\begin{itemize}
\item (T1) $\emptyset \in \mathscr{T}, X \in \mathscr{T}$.
\item (T2) The union of \textbf{any} members of $\mathscr{J}$ is a member of $\mathscr{T}$.
\item (T3) The intersection of \textbf{finitely} many members of $\mathscr{T}$ is a member of it.
\end{itemize}
\end{rema}

\begin{defi}
{Continuous mapping}
For $X=(X,d)~and~Y=(Y,\tilde{d})$, $T: X \longrightarrow Y$ is continuous at point $x_0$ if for every $\epsilon >0$ there's a $\delta >0$ such that
\end{defi}

\begin{theo} 
{Continuous mapping}
A mapping $T: X \longrightarrow Y$ is continuous iff \textbf{the inverse image of any open subset of Y is an open subset of X}.
\end{theo}

\begin{defi}{Dense set, separable space}
        A subset M of a metric space X is \textit{dense} in X if
        \begin{equation*}
                \bar{M}=X
        \end{equation*}
        X is \textit{separable} if it has a \textbf{countable} subset which is dense in X.
\end{defi}
\\
\textbf{Examples}
\book{37}

\subsection{Convergence, Cauchy Sequence, Completeness}
\begin{defi}{Convergence of a sequence, limit}
\end{defi}

\begin{lemm}{Boundedness, limit}
        (a) A convergent sequence in X is bounded and its limit is unique. \\
        (b) If $x_n \rightarrow x$ and $y_n \rightarrow y$ in X, then $d(x_n, y_n) \rightarrow d(x,y)$
\end{lemm}.
\end{document}
%\includepdf[link=true,pages=-,fitpaper]{./papers/<++>.pdf}
