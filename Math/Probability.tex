\documentclass{article}
\usepackage{amsmath}
\usepackage{amssymb}
\usepackage{dsfont}
\usepackage{unicode-math}
\numberwithin{equation}{subsection}

\usepackage{pdfpages}
\usepackage{verbatim}
\usepackage{xeCJK}
\usepackage{hyperref}
%\usepackage{apacite}
\usepackage{natbib}
\usepackage{grffile}
\usepackage{listings}
\lstset{
 columns=fixed,       
 numbers=left,                                        % 在左侧显示行号
 numberstyle=\tiny\color{gray},                       % 设定行号格式
 frame=single,                                          % 不显示背景边框
 keywordstyle=\color[RGB]{40,40,255},                 % 设定关键字颜色
 numberstyle=\footnotesize\color{darkgray},           
 commentstyle=\it\color[RGB]{0,96,96},                % 设置代码注释的格式
 stringstyle=\rmfamily\slshape\color[RGB]{128,0,0},   % 设置字符串格式
 showstringspaces=false,                              % 不显示字符串中的空格
 language=c++,                                        % 设置语言
}
\hypersetup{
    colorlinks=true,
    linkcolor=blue,
    filecolor=magenta,
    urlcolor=cyan,
}
%--------Commands for local pdf reference------%
\newcommand{\citeint}[2]{\cite{#1}\hyperlink{./references/#1.pdf.#2}{(P#2)}}
\newcommand{\citeext}[2]{\href[page=#2]{./references/#1.pdf}{(P#2)}}
\newcommand{\citeinclude}[1]{\includepdf[link=true,pages=-,fitpaper=true]{./references/#1.pdf}}
%--------Environments for math------------%
\newcounter{topic}[subsection]
\newenvironment{defi}{\refstepcounter{topic}\label{\thesubsection.\thetopic}\par\medskip
   \noindent \textbf{Definition~\thesubsection.\thetopic~} \rmfamily }{ \medskip}
\newenvironment{theo}{\refstepcounter{topic}\label{\thesubsection.\thetopic}\par\medskip
   \noindent \textbf{Theorem~\thesubsection.\thetopic~} \rmfamily }{\medskip}
\newenvironment{coro}{\refstepcounter{topic}\label{\thesubsection.\thetopic}\par\medskip
   \noindent \textbf{Corollary~\thesubsection.\thetopic~} \rmfamily }{ \medskip}
\newenvironment{lemm}{\refstepcounter{topic}\label{\thesubsection.\thetopic}\par\medskip
   \noindent \textbf{Lemma~\thesubsection.\thetopic~} \rmfamily }{ \medskip}
\newenvironment{exam}[1]{\refstepcounter{topic}\label{\thesubsection.\thetopic}\par\medskip
   \noindent \textbf{Example~\thesubsection.\thetopic~} \rmfamily }{ \medskip}

\newenvironment{rema}[1]{\refstepcounter{topic}\label{\thesubsection.\thetopic}\par\medskip
   \noindent \textbf{Remark~\thesubsection.\thetopic~} \rmfamily }{ \medskip}
\newenvironment{prof}{\par\medskip \noindent \textit{Proof.}~\rmfamily}{\medskip}
\title{Overview of Reinforcement Learning}
\author{陈辉}
\date{}
\begin{document}
\maketitle
\tableofcontents
\newpage
\section{Measure Theory}
\subsection{Probability Spaces}
\begin{defi}
        A \textbf{Probability space} is a triple $(\omega, \mathcal{F}, P)$. Where $\mathcal{F}$ is a $\sigma$-\textbf{field}, that is, a nonempty collection of subsets of $\omega$ that satisfy:\\
        (i) if $A \in \mathcal{F}$ then $A^c \in \mathcal{F}$\\
        (ii) if $A_i \in \mathcal{F}$ is a countable sequence of sets then $\cup_i A_i \in \mathcal{F}$\\
        Without P, $(\omega, \mathcal{F})$ is a \textbf{measurable space}, that we can define a nonnegative countably additive set function $\mu : \mathcal{F} \rightarrow R$ with:\\
        (i) $\mu(A) \geq \mu(\emptyset) = 0$ for all $A \in \mathcal{F}$.\\
        (ii) if $A_i \in \mathcal{F}$ is a countable sequence of disjiont sets, then
        \begin{equation*}
                \mu(\cup_i A_i) = \sum_i \mu(A_i)
        \end{equation*}
\end{defi}






\setcounter{section}{5}
\section{Markov Chains}
\subsection{Definitions}




\bibliographystyle{plain}
%\bibliography{overview_RL}

%\citeinclude{Playing_Atari_with_Deep_Reinforcement_Learning}
%\citeinclude{Asynchronous_Methods_for_Deep_Reinforcement_Learning}
\end{document}


%\includepdf[link=true,pages=-,fitpaper,angle=-90]{./papers/<++>.pdf}
